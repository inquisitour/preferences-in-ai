\documentclass[11pt]{article}
\usepackage[utf8]{inputenc}
\usepackage{geometry}
\usepackage{booktabs}
\usepackage{graphicx}
\usepackage{hyperref}

\title{Preferences in AI Project Report}
\author{Pratik Deshmukh}
\date{\today}

\begin{document}

\maketitle

\section{Introduction}
This report presents experiments on free-riding in sequential decision-making
under different \emph{statistical cultures}, following the framework of
\cite{nardi2022}.

\section{Methodology}
We repeat the experiments from the paper but use several different statistical
cultures: impartial culture (p-IC), disjoint groups, and the $(p, \phi)$-resampling
model. For each, we run multiple seeds and compare welfare and risk metrics under
sequential utilitarian and seq-PAV rules.

\section{Results}
The combined results table is automatically generated by the experiment
pipeline. The table below is included directly from the output file:

\begin{table}[h!]
\centering
\resizebox{\textwidth}{!}{%
\input{tables/combined.tex}
}
\caption{Combined results across cultures and rules. Welfare metrics are
(utilitarian, egalitarian, Nash), while risk metrics include success and harm
rates.}
\label{tab:combined}
\end{table}

\section{Discussion}
We observe that the choice of statistical culture significantly affects both
welfare and manipulation risks. For example, disjoint cultures show notably
lower welfare but also lower manipulation success rates, while resampling tends
to yield higher welfare but with moderate manipulation risk.

\section{Conclusion}
These experiments confirm that statistical cultures strongly influence the
perceived robustness of voting rules. Future work may explore richer parameter
grids and robustness to noise.

\bibliographystyle{plain}
\bibliography{references}

\section*{Repository}
The full project code and report sources are available at: \\
\href{https://github.com/inquisitour/preferences-in-ai-project}{github.com/inquisitour/preferences-in-ai-project}

\end{document}
